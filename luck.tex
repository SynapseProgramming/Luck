
\documentclass[11pt]{article}
\usepackage{graphicx}
\usepackage[english]{babel}
\usepackage{float}
\usepackage{amsmath}
\usepackage[section]{placeins}
\usepackage{hyperref}
\graphicspath{ {./images/} }

\usepackage[T1]{fontenc}
\usepackage{listings}
\usepackage{hyperref}
\usepackage{color}

\definecolor{dkgreen}{rgb}{0,0.6,0}
\definecolor{gray}{rgb}{0.5,0.5,0.5}
\definecolor{mauve}{rgb}{0.58,0,0.82}

\lstset{frame=tb,
  language=C++,
  aboveskip=3mm,
  belowskip=3mm,
  showstringspaces=false,
  columns=flexible,
  basicstyle={\small\ttfamily},
  numbers=none,
  numberstyle=\tiny\color{gray},
  keywordstyle=\color{blue},
  commentstyle=\color{dkgreen},
  stringstyle=\color{mauve},
  breaklines=true,
  breakatwhitespace=true,
  tabsize=3
}

\setlength{\parindent}{0pt}
\setlength{\parskip}{0pt plus 0.5ex}
%% adjust spacing for all itemize/enumerate

\begin{document}


\tableofcontents

\section{Luck and Evolution}
\subsection{Types of Selection}
The general case for all three cases of selection is the non-random survival of randomly hereditary equipment.

In artificial selection, humans  choose characteristics that they desire.  Thus these characteristics survive non-randomly.  The reason that these characteristics came to be so good is 	RANDOM.

Similarly for sexual and natural selection.

\begin{enumerate}
 \item {
       \textbf{Artificial Selection}
       \\ Humans (as a selecting agent) force an increase or decrease in the frequency of specific genes in the gene pool of a particular species
       E.g. (Domesticated) dogs were created by selecting certain characteristics of wolves (e.g. more docile)
       E.g. domesticated plants.
       Beautiful plants
       }
 \item {
       \textbf{Sexual Selection}
       \\ When a female of the species observes a trait they desire, they are able to choose that particular mate, and the good trait eventually gets passed down by reproduction.
       Examples: lions with beautiful manes, peacocks with large plumes, birds of paradise with good dancing skills.
       }
 \item  {
       \textbf{Natural Selection}
       \\ Certain traits are more favorable in certain environments, which would lead to better odds of surviving. Those that survive can then pass down their genes, which increases the frequency of specific traits that allow for better survival.
       An example here would be the long necks in giraffes, allowing them to get to leaves that other giraffes cannot reach, securing their food supply.
       }
\end{enumerate}
\subsection{Slow Process of Evolution}
Evolution is a slow process as it goes slowly by accumulation of mutations across many generations/across long periods of time
For most species, evolution is a slow process as it takes many generations for mutations to accumulate and create something complicated, such as an eye or a wing.  rare events of mutations -> takes a long period of time
\subsection{Evolution by natural selection criteria (nurse 2020)}
\begin{itemize}
 \item {
       First, they must be able to reproduce.
       }
 \item {
       Second, they must have a hereditary system, whereby information defining the
       characteristics of the organism is copied and inherited during their reproduction.
       }
       \item{
                   Third, the hereditary system must exhibit variability, and this variability must be
                   inherited during the reproductive process. It is this variability that natural selection
                   operates upon. It transforms a slow and randomly generated source of variability into the
                   apparently boundless and constantly changing range of life forms that flourish around us.
                   Additionally, for this to work efficiently, living organisms must die. That way, the next
                   generation, potentially containing genetic variants that give them a competitive edge, can
                   replace them
                   
                   
             }
\end{itemize}

\section{Luck and Circumstance at birth}
\subsection{People dont rise from nothing}
We do owe something to parentage and patronage. The people who stand before kings may look like they did it all by themselves. But, in face they are invariably the beneficiaries of hidden advantages and extraordinary opportunities and cultural legacies that allow them to learn and work hard and make sense of the world in ways others cannot. (gladwell, M 2008 pg 21)
\subsection{the Matthew effect}
It is those who are successful, in other words, who are most likely to be given the kinds of speical opportunities that lead to further success. It's the rich who get the biggest tax breaks. It's the best students who get the best teachers and most attention. And it's the biggest nine and 10 year olds who get the most coaching and practice. Success is the result of what sociologists like to call an accumulated advantage. The professional hockey players starts out a little bit bigger, and that edge in turn leads to another opportunity, which makes the initially small difference bigger still and so on and so on until the hockey player is a genuine outlier.

\subsection{Relative Age phenomenon}
In any elite group of hockey players, 40 percent of the players will have been born between January and March, 30 percent between April and June, 20 percent between July and September, and 10 percent between October and December (gladwell, M 2008 pg 21)
\\The relative age phenomenon is the skill difference within a grouping of children by calender year between children born in the earlier part of the year and later part of the year, due to cutoff dates which separate different ages
At a young age, a difference in say, 10 months, could make a huge difference in physical maturity (Hockey league, outliers p21; about education, outliers p30\&31)
Soln: Outliers p 35? Separate leagues (two leagues for same calender year, one first half one second half); for schools, breaking them down by month of birth (outliser, p36)
Singapore parallels: Gifted Ed test
\subsection{Effects of country of birth}
Yes, it makes a difference which rich country I’ll migrate to.
It is preferable to migrate to a rich, egalitarian country e.g. Sweden which provides adequate welfare benefits to its residents (offer more benefits to the poor than the rich) . That country should have a stronger salary opportunity (after accounting for the exchange rate) as compared to my home country. Preferably a country where the citizens are not too hostile of immigrant workers and without many restrictions to entry in terms of qualifications.
Countries with similar culture to the home country may be preferable to smooth the process of assimilation into the new country.
Sweden has a high GST 25\% which is imposed on the local to support the immigrants and that creates tension between the local and immigrants.

If you choose to go to a country with a lot of income inequality, it is unlikely you can reap most of the benefits of migration (e.g. Brazil). Citizenship Premium at the bottom is greater than at the top when poor unequal VS rich equal. Repercussions for that country is that they might attract people with low talent who are lured in by the benefits given to the people of the country.
\subsection{Effects of immigration}


Rich country:
Assuming people in rich countries would not want low paying low skill jobs, restricting immigrants to only the qualified and the rich would result in a lack of manpower for low skilled jobs.

Poor country:
The country faces a loss of talent and individuals that were the driving forces behind the country’s economic growth. The other citizens in the country now have a greater burden.


The problem is that from the global perspective, this approach to migration is heavily discriminatory. To one set of “discriminations,” the citizenship rent, we add another set of discriminations whereby this rent may also be enjoyed by those who were not lucky enough to have been born in a rich country but have exceptional abilities or wealth. We run the risk that such policies will result in the poor world, and I am thinking especially of Africa here, becoming even poorer as its most educated and wealthiest members leave. (Milanovic, 2016, p. 136)
\section{Luck and Social Mobility}
\subsection{Rat Race equilibrium}

Definition of quality of life: educational attainment, physical and mental well-being, access to basic necessities

Rat-race equilibrium: ‘mechanics of meritocratic production put additional upward pressure on elite industry, driving superordinate workers  to yield ever more intense effort and ever-longer hours — more than anyone actually wants’

Elite workers today are almost expected not to have personal lives. (Meritocracy Trap, 2019, Page 190) One elite firm, concerned that its employees were working too hard, granted unlimited vacation time, this triggered a reduction in vacation actually taken.

‘Happy guy in college’ who ‘became “a snappy … really uncomfortable guy to be around’

Elite workers want to prove that they are hard working via staying long hours in the office, others follow so that they don’t look bad in the eyes of their bosses. Even when a company gives unlimited leaves, employees would rather take fewer leaves to prove their loyalty to the company. This leads to high stress levels and several health issues, while feeling lesser sense of fulfillment.
\subsection{Meritocracies effects on social mobility}
\subsubsection{Against Meritocracy}
\begin{itemize}
 \item {
       \textbf{  Deaths of Despair and the Future of Capitalism:}
       \\    The wealthy can pay for more, and higher-quality, coaching for college entrance exams and essays, as well as for diagnoses of disabilities that allow their children extra time for classwork and exams.13  (deaths \& despair).
       Wealthy parents paid bribes to secure a place for their children in elite colleges.
       }
 \item {
       Jobs that were once opened to nongraduate are now reserved for those with a college degree. This means that the less fortunates that could not afford a college degree will have lesser jobs opportunities.
       }
 \item {
       \textbf{The Meritocracy Trap:}
       \\ Pg 182: Rich families run the gauntlet of elite education to give their children the exceptional training and skills required to get and to do superordinate jobs, so that they might land on the right side of the meritocratic divide.
       }
 \item {
       \textbf{Elite Training Succeeds}
       \\ The investments that rich families make in their children’s human capital pay off. Children from the richest fifth of households are roughly seven times more likely than children from the poorest fifth to end up in the top quintile of the income distribution as adults, roughly nine times more likely to end up in the top quintile of the wealth distribution, and roughly twelve times more likely to end up in the top quintile of the education distribution.
       
       Education is a prime sorting mechanism. The median college graduate will make more money over his lifetime than 93\% of workers without a high school degree and than 86\% of workers with a high school degree only; and the median professional school graduate will make more money than nearly 99\% of high school dropouts.
       
       Graduate and especially professional degrees yield a greater income premium.
       }
 \item {
       \textbf{The tyranny of merit: }
       \\ Pg 178: This meritocratic arms race tilts the competition in the favor of the wealthy and enables the affluent parents to pass on the privilege on to their kids.
       
       Prosperous parents are able to give their kids a powerful boost in the bid for admission to elite colleges.
       
       First, my having this or that talent is not my doing but a matter of good luck and I do not merit or deserve the benefits that derive from luck.
       
       Second, that i live in a society that prizes the talents i happen to have is also something for which i can claim credit. This too is a matter of good fortune.
       
       It is a flawed assumption to consider our success and failure is solely based on our merit as this assumes that economy is a field of fair competition, untainted by privilege or prejudice.
       }
       \item{
                   \textbf{Aristocracy of talent pg 6}
                   \\           There is a problem with calcification: the same family names crop up in lists of top scholarship winners and office-holders, starting with the Lee family itself.
                   
                   Conformity is more important than originality.
                   
                   It combines meritocracy with an intrusive, and somewhat fussy, authoritarianism.
             }
\end{itemize}
\subsubsection{Support Meritocracy}
\begin{itemize}
 \item {
       \textbf{The Aristrocracy of Talent (reading)}
       The meritocratic idea made the modern world, sweeping aside race-and sexbased barriers to competition, building ladders of opportunity from the bottom of society to the top, and electrifying sluggish institutions with intelligence and energy. \textbf{Discrimination on the basis of race and sex is now illegal across the advanced world.} Women take up more than half of the places in most Western (and in many emerging-country) universities. Kamala Harris, a woman of Jamaican and Indian heritage, is vice-president of the United States, and may well follow Barack Obama to the Oval Office. None of that would have been possible without the meritocratic idea. (Wooldridge, 2021)
       }
       \item{
                   \textbf{Reduces Race/gender descrmination}
                   
                   Meritocracy succeeds because it does a better job than the alternatives of reconciling the two great tensions at the heart of modernity: between efficiency and fairness on the one hand, and between moral equality and social differentiation on the other. It screens job applicants for competence. Vaccines save our lives rather than poisoning us because highly trained scientists develop them and other highly trained scientists test and regulate them. But, at the same time, meritocracy gives everybody a chance to put their name into the sorting hat.
             }
       \item{
                   \textbf{Pick the best people for the job in cases where failure is not an option}
                   \\ They demonstrate that countries that favour recruiting professional managers through open competition have higher growth rates than those that favour recruiting amateur managers through personal connections.
             }
       \item{
                   \textbf{Gifted Education Program}
                   \\ The city-state is much concerned with identifying children at the very top of the ability range. All children are assessed at the age of eight or nine in maths, English and reasoning. The top 1 per cent are transferred into a Gifted Education Programme which is run in nine primary schools up to the age of twelve. They can then choose if they want to go to certain secondary schools that also offer a Gifted Programme. These selected children are given ‘personalized education plans’ that include extra teaching in some subjects, advanced placement in some classes and access to self-taught online courses.
             }
 \item {
       \textbf{Better standard of living, longer life expectancy}
       \\ Two recent studies are particularly telling. Four economists at the University of Chicago’s Booth School of Business have examined America’s GDP growth per person in 1960–2010 in the light of the distribution of talent. They claim that roughly a fifth of that growth can be explained by the improved allocation of talent, particularly the opening of highly skilled professions to new talent pools. In 1960, 94 per cent of America’s doctors and lawyers were white men. By 2010, that number had shrunk to 60 per cent.5 That makes for both a more productive and a more just society
       }
\end{itemize}
\section{Luck and Free Will}
\subsection{Chaos Theory \& Quantum Mechanics}
\begin{itemize}
 \item {
       \textbf{Chaos Theory}
       \\ Behavior of complex systems cannot be accurately predicted for as it is impossible to account for the multitude of factors that affect outcomes. Thus, small changes will have cataclysmic effects on the final result. This complexity makes prediction about chaotic systems extremely difficult. However, \textbf{chaos theory is still deterministic} as the systems are hard to predict, but is not random and can thus still be defined.
       }
 \item {
       \textbf{Quantum Mechanics}
       \\\textbf{TLDR:} quantum mechanics does not support free will, as randomness does not imply free will.
       \\ Quantum systems are inherently unpredictable. It works on a particle level, which makes it impossible to predict eventual outcomes. However, we can calculate the probability of each event happening. It seems that the window between the probabilities and the eventual outcome is perfect for free will, with agents freely deciding what the eventual outcome will be. However, the eventual outcomes in quantum mechanics are decided by randomness. And \textbf{random decisions cannot be freely-made decisions.}  The quantum equations lay out many possible futures, but they deterministically chisel the likelihood of each in mathematical stone. Thus, it is still bound to the laws of physics and math.
       
       \begin{enumerate}
        \item {
              ‘Until the End of Time’ pg 148 and 149: \\ But in quantum physics, as we have seen, the equations predict only the likelihood of how things will be at any future moment. By inserting an element of  probability-chance-quantum mechanics seems to provide a modern and experimentally motivated version of the Epicurean swerve, slackening the deterministic reins.
              }
        \item {
              Quantum mechanics takes as input the state of the world now and produces a unique table of probabilities for the state of the world tomorrow. The quantum equations lay out many possible futures, but they deterministically chisel the likelihood of each in mathematical stone. Much like Newton, Schrodinger leaves no room for free will.
              }
        \item {
              Consider an electron that according to quantum mechanics has a 50\% chance of being here and a 50\% chance of being there. You cannot freely pick the outcome. This attests to the outcome being random, and random outcomes are not freely willed choices. The results have a statistical regularity. A freely willed choice is not constrained, even in a statistical sense, by mathematical rules.
              }
       \end{enumerate}
       }
\end{itemize}
\subsection{Free Will does not exist}

We need to recognize that although the sensation of free will is real, the capacity to exert free will--the capacity for the human mind to transcend the laws that control physical progression--is not.
(Greene, 2020, p.158)
\begin{itemize}
 \item {Our choices are the result of our particles coursing one way or another through our brains. Our actions are the result of our particles moving this way or that through our bodies. And all particle motion--whether in a brain, a body, or a baseball--is controlled by physics and so is fully dictated by mathematical decree. (Greene, 2020, p.147)
       }
       \item{
                   A freely willed choice is not constrained, even in a statistical sense, by mathematical rules. But as the evidence demonstrates in this instance and all others too, the math does rule. So although the passage from quantum probabilities to experiential certainties remains puzzling, it is clear that free will is not part of the process. To be free requires that we are not marionettes whose strings are pulled by physical law. (Greene, 2020, p.149)
             }
       \item{
                   might free will be lurking in the answer? Unfortunately, no. Consider an electron that according to quantum mechanics has a 50 percent chance of being here and a 50 percent chance of being there. Can you freely pick the outcome-here or there--that an observation of its position will reveal? You can't.
                   (Greene, 2020, p.149)
             }
 \item {
       To sum up: We are physical beings made of large collections of particles governed by nature's laws. Everything we do and everything we think amounts to motions of those particles.
       Our choices seem free because we do not witness nature's laws acting in their most fundamental guise;
       (Greene, 2020, p.150)
       
       
       
       }
\end{itemize}

\end{document}
